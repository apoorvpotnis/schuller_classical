\documentclass[a4 paper, oneside, 12pt]{book}

\usepackage[ngerman, italian, english]{babel}
\PassOptionsToPackage{math-style=ISO, bold-style=ISO, sans-style=italic, nabla=upright, partial=upright, warnings-off={mathtools-colon,mathtools-overbracket}}{unicode-math}
\usepackage{mathtools}
\usepackage{microtype}
\usepackage{imakeidx}
\makeindex[intoc]
\usepackage{setspace}
\usepackage[style=british]{csquotes}
\usepackage[default, newcmbb]{fontsetup}
\usepackage{amsthm}
\theoremstyle{definition}
\newtheorem{thm}{Theorem}
\newtheorem{defn}{Definition}

\begin{filecontents}{schuller_classical.bib}
	@book{Arnold,
		author = {Arnold, Vladimir Igorevich},
		edition = 2,
		isbn = {978-1-4419-3087-3},
		publisher = {Springer Science+Business New York},
		series = {Graduate Texts in Mathematics 60},
		title = {Mathematical Methods of Classical Mechanics},
		year = {1989},
		addendum = {Translated from Russian by K.\ Vogtmann and A.\ Weinstein.}
	}

	@book{Moretti,
		author = {Moretti, Valter},
		isbn = {978-3-031-27611-8},
		publisher = {Springer Nature Switzerland AG},
		series = {\selectlanguage{italian}La Matematica per il 3+2\selectlanguage{english}},
		title = {Analytical Mechanics: Classical, Lagrangian and Hamiltonian Mechanics, Stability Theory, Special Relativity},
		year = {2023},
		addendum = {Translated from Italian by Simon G.\ Choissi}
	}
\end{filecontents}

\usepackage[sorting=none]{biblatex}
\addbibresource{schuller_classical.bib}

\usepackage{embedall}
\embedfile[desc = bibliography source file]{schuller_classical.bib}
% \embedfile{schuller_classical.tex}

\usepackage{hyperref}
\hypersetup{citecolor=red, pdfencoding=auto, psdextra, colorlinks=true, linkcolor=red, breaklinks=true, urlcolor=blue, pdftitle={Schuller's Lectures on Classical Mechanics}, bookmarksopen=true, pdfauthor={Apoorv Potnis}, pdfsubject={Schuller's Lectures on Classical Mechanics}, unicode=true, pdftoolbar=true, pdfmenubar=true, pdfstartview={FitH}, pdfkeywords={Frederic Schuller, Classical Mechanics, Differential Geometry, Calculus of Variations, Mathematical Physics, Lecture Notes}}
\usepackage{cleveref, xurl}
\usepackage[numbered]{bookmark}

\newcommand{\ltwo}{\mathup{L\kern-0.5pt^2}}
\newcommand{\position}{\mathup{Q}}
\newcommand{\momentum}{\mathup{P}}
\newcommand{\rthree}{\mathbb{R}^3}
\newcommand{\rr}{\mathbb{R}}
\newcommand{\cc}{\mathbb{C}}
\newcommand{\nn}{\mathbb{N}_0}
\newcommand{\dirac}{\symup{\delta}}
\renewcommand*{\hbar}{\mathrm{^^^^0127}}
\renewcommand{\i}{\mathrm{i}}
\newcommand{\e}{\mathrm{e}}
\newcommand{\cinfinity}{\mathrm{C}^\infty}
\newcommand{\domain}{\mathcal{D}}
\newcommand{\identity}{\mathrm{id}}
\DeclarePairedDelimiter{\norm}{\lVert}{\rVert}
\DeclarePairedDelimiter{\abs}{\lvert}{\rvert}
\newcommand{\der}{\operatorname{d\!}{}}

\title{Schuller's Lectures on Classical Mechanics}
\author{Frederic Schuller\\Apoorv Potnis}
\date{\today}

\begin{document}
	\hypertarget{TitlePage}{}
	\bookmark[dest=TitlePage]{Title Page}
	\maketitle

	\chapter*{Preface}
	\hypertarget{Preface}{}
	\bookmark[dest=Preface]{Preface}
	These are lecture notes by Apoorv Potnis of the lecture series `\selectlanguage{ngerman}Theoretische Physik 1: Mechanik\selectlanguage{english}' (Theoretical Physics 1: Mechanics), given by \textbf{Prof.\ Frederic Paul Schuller} in 2014 at the \selectlanguage{ngerman}Friedrich-Alexander-Universität Erlangen-Nürnberg\selectlanguage{english}. Prof.\ Schuller discusses classical mechanics in a mathematically rigorous fashion in this course. While the original lecture series is in German, these notes are in English and have been prepared using YouTube's automatic subtitle translation tool. The video lecture series is available at \url{https://www.youtube.com/watch?v=FNJOyxOp3Ik&list=PLPO5pgr_frzTeqa_thbltYjyw8F9ehw7v&index=} and at \url{https://www.fau.tv/clip/id/4301}.
	\clearpage

	\hypertarget{Contents}{}
	\bookmark[dest=Contents]{Contents}
	\tableofcontents

	\chapter{A Bird's Eye View of Physics}

	What is the goal of physics? According to Prof.\ Schuller, the only goal of physics is to predict the future, no more and no less. What is the goal of theoretical physics? Prof.\ Schuller remembers Wittgenstein to remark that the goal is to say all that can be said clearly. But in order to say things clearly, one needs the language of mathematics. Yet another test.

	\chapter{Topological Manifolds}

	\chapter{Differentiable Manifolds}

	\chapter{Tangent Spaces}

	\chapter{Tensors and Tensor Fields}

	\nocite{*}
	\printbibliography[heading=bibintoc]
	\printindex
\end{document}
